\documentclass[10pt,a4paper,twocolumn]{article}
\usepackage[utf8]{inputenc}

\begin{document}

\title{Concurrent Programming Languages}

\author{
Alexis Beingessner
\and
Troy Hildebrandt
}

\maketitle

\section{Abstract}

Developing applications for distributed systems provides unique challenges that
many modern programming languages are not inherently equipped for. These
challenges are often ones that arise with parallel and concurrent programming,
with additional concerns such as node and network failures, software
survivability, and network latency issues, among others.

Certain languages attempt to tackle some of the major problems associated with
developing distributed applications. These languages provide built-in facilities
for creating nodes that communicate over a network, error handling, message
passing, and mechanisms for more transparently simulating shared memory. This
simplifies fault-tolerance for synchronous access to data across multiple
machines.

We look at several languages and their respective extensions that enable
distributed application development, including Erlang, Concurrent/Cloud Haskell,
and Ada. For each, we examine their strengths and innovations, and assess their
impact on distributed computing. The languages explored all provide unique
solutions to some of distributed programming's toughest problems, in some cases
using completely different programming paradigms.

\section{Introduction}

A system can generally be implemented to run in four basic ways:
sequentially, concurrently, in parallel, or distributed. 

In a sequential system steps logically execute in a fixed order. Sequential
systems have long dominated the programming landscape as easy to implement and
reason about.  that dominate the programming landscape. However for many
problems a sequential implementation has undesirable performance or design
implications. If multiple tasks need to be performed, steps require particular
real-world resources, or must grapple with the non-sequential nature of the real
world, then sequential operation can be far from optimal.

Concurrent programming involves allowing some operations to logically occur in
arbitrary, potentially overlapping, order. For instance, it is generally
desirable to be able to handle multiple TCP connections, even if one does not
have the resources to actively perform work for them all. Even if only one
connection is actually being worked on at a time, allowing each connection to be
handled concurrently allows us to perform productive work \emph{more often} than
sequential connection handling: if one connection is blocked on waiting for a
response or system resource, another connection can be worked on. Concurrency
can be implemented on top of a logically sequential system using time slicing,
though this limits the performance that can be realized. CPU-bound applications
will see little to no benefit from such a system.

Parallel programming involves actually performing operations at the same time.
Parallel programming is most easily enabled by concurrent programming. If a
program is already concurrent, then concurrent tasks can simply be driven
forward at once. A sequential system can be built on top of a parallel system
through synchronization mechanisms (simplifying reasoning while reducing the
value of the parallel system), but a parallel system cannot be built on top of a
sequential one.

Distributed programming involves actually performing operations across several
otherwise independent machines. This enables computation to be done at a greater
scale than a single machine could ever hope to do with current technology. It
also allows applications to be made robust against hardware failures, while
ironically increasing the probability of such a failure. Single-machine
applications can often ignore the consequences of hardware failures as unlikely
or at worst properly handled by the underlying operating system. However the
ultimate consequence of no longer running can itself be an unacceptable.

Distributed computing can itself be decomposed into two sub-domains: cluster
computing and global computing. The distinction largely being exactly how far
apart the systems are. Clusters are co-located, enabling faster communication
and tighter control over the system. Global systems are distributed across the
globe, increasing the cost of intra-communication while making the system more
available to third parties and more robust to local disasters. For the purposes
of this work, these distinctions will not be particularly important.

Most hardware, operating systems, and programming languages readily permit an
obvious sequential implementation. In particular C was developed in a world of
single-process, single-user, CPU-bound systems, and its highly sequential and
computation-oriented nature reflects this. However modern hardware
and operating systems are highly parallel and significantly more powerful.
Applications are expected to serve millions of logically concurrent and IO-bound
requests on datasets orders of magnitude larger than a single system could
handle. These problems \emph{necessitate} a distributed system. While it's not
\emph{impossible} to get C to efficiently handle these problems, it's certainly
error-prone and difficult. For this reason we argue that C is simply
\emph{inappropriate} for distributed systems.

But C is \emph{the} systems programming language. Its only real competition is
an increasingly complex pseudo-superset of itself (C++). Therefore we are left
asking what programming languages \emph{are} appropriate for distributed systems
to be implemented in.

This paper surveys the programming language literature for how different
languages enable efficient and correct distributed systems to be more easily
developed. In particular we observe that functional languages with support for
\emph{concurrent} programming allow systems to be more trivially parallelized.
When combined with an emphasis on message passing over shared state, tasks can
then be made distributed without significant semantic changes over a single
system deployment. In fact, one can deploy and debug a distributed system on a
single machine accurately. However message passing alone is insufficient.
Communication and task handling must be done in a fault-tolerant manner. Tasks
cannot be assumed to receive a message or even consistently \emph{exist}.

\section{Erlang}

Erlang is a primarily functional language that was designed to solve the
problems faced by the telecom industry. Telecom systems are expected to meet the
following requirements \cite{dacker2000concurrent}:

\begin{itemize}
    \item The system must be able to handle very large numbers of concurrent activities.
    \item Actions must be performed at a certain point in time or within a certain time.
    \item Systems may be distributed over several computers.
    \item The system is used to control hardware.
    \item The software systems are very large.
    \item The systems should be in continuous operation for many years.
    \item Software maintenance (reconfiguration, etc) should be performed without stopping the system.
    \item There are stringent quality, and reliability requirements.
    \item Fault tolerance both to hardware failures, and software errors, must be provided.
\end{itemize}

Erlang's primary tool for handling these problems is its \emph{process} system.
\cite{erlangthesis} An Erlang process is a lightweight task that is logically
isolated from all other processes. They run concurrently, and can only
communicate by passing messages to processes whose secret ids they know. An
Erlang program is intended to be decomposed into many small processes. Further,
processes may be broken up into a hierarchy of quality to be attempted in a
best-effort manner. The hardest and best process is always preferred, but if
that fails an easier but less-good process is tried instead. This helps ensure
that even in face of hardware and software errors, the system is able to
maintain some basic level of operation.

The model of computation for \emph{individual} processes is that they're
expected to try to do the right thing, but are considered unreliable. One should
be prepared for tasks to experience an error or disappear completely at any
time. Messages sent to a task cannot be assumed to be received; a response must
be explicitly sent back to confirm that a message was sent and handled
successfully. Erlang's processes system does its best to ensure that a flawed
process cannot harm the execution of a correct process unless the correct one
specifically opts into such a dependency.

Processes can also have their code updated at runtime. This can be done fairly
cleanly due to the functional nature of Erlang: loops are generally encoded as
recursion, and so any code that is run is likely to be re-entered by a function
call which can be redirected to the new version. It is otherwise the
programmer's responsibility to ensure that new code is compatible with the old
running code. \cite{erlangthesis}

Erlang's processes enable it to reasonably meet all the requirements for a
telecom system. As such it saw wide deployment at Ericsson and Nortel. It
famously empowered the Ericsson AXD301 switching system to obtain 9 9's
(99.9999999\%) of reliability on an 11 node deployment; effectively 0 down time.
\cite{erlangthesis}

%TODO: dig into concrete deployments more
%TODO: dig into purity ("clean" vs "dirty" code)?
%TODO: elaborate on "let it crash" model of computation?



\section{Cloud Haskell}

From Microsoft Research comes Cloud Haskell, an extension to the Haskell
programming language that attempts to solve many of the problems with creating
distributed applications. XXX : More

In any distributed system, there must be some level of communication between
processes running on many different machines, and one of the most costly
operations in such a system is the transfer of data between nodes when
interprocess communication is necessary. Cloud Haskell employs a message passing
system inspired by Erlang wherein no processes have access to each others' data,
and this information must be explicitly sent between processes if they are to
share it.

Use of the Erlang message passing method here is beneficial for a variety of
reasons; if messages must be explicitly passed between processes, there is an
explicit cost to the communication which may drive developers to rely less on
interprocess communication and look to structuring their computation differently
to minimize costly message passing. When processes have to send messages before
data is shared, it is also possible to avoid the corruption or contamination of
data in one process by another, ensuring that failure in one process does not
result in the failure of another process.

What Cloud Haskell introduces that Erlang doesn't is the fact that Cloud Haskell
is built on Haskell, a purely functional programming language.



\section{Ada}



\section{Conclusion}





\addcontentsline{toc}{chapter}{References}
\small

\bibliographystyle{abbrv}
\bibliography{bibliography}

\end{document}
